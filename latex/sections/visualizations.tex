\chapter{Trực quan hóa}

% \section{Biểu đồ hiệu suất}

% \subsection{Độ trễ (Latency)}
% Độ trễ được đo bằng thời gian trung bình để thực hiện một truy vấn, tính bằng micro giây (µs). Kết quả từ thí nghiệm benchmark được trình bày trong \Cref{tab:latency_comparison}.

% \begin{table}[htbp]
%     \centering
%     \caption{So sánh độ trễ giữa HNSW và Brute-force}
%     \label{tab:latency_comparison}
%     \begin{tabular}{@{}cccccc@{}}
%         \toprule
%         \textbf{K} & \textbf{Brute-force (µs)} & \textbf{HNSWlib (µs)} & \textbf{HNSW-Flat (µs)} & \textbf{HNSW-SQ (µs)} & \textbf{HNSW-PQ (µs)} \\
%         \midrule
%         1   & 587.3     & 260.7   & 213.0  & 208.8  & 98.3   \\ 
%         5   & 593.9     & 261.7   & 220.0  & 208.1  & 98.7   \\
%         10  & 600.9     & 266.9   & 218.8  & 211.0  & 100.0  \\
%         20  & 591.6     & 258.9   & 214.3  & 211.2  & 102.6  \\
%         50  & 593.7     & 264.9   & 216.9  & 219.8  & 109.1  \\
%         100 & 573.9     & 267.4   & 229.3  & 230.9  & 119.6  \\

%         \bottomrule
%     \end{tabular}
% \end{table}

% \begin{figure}[htbp]
%     \centering
%     \includegraphics[width=1\textwidth]{images/time_recallK.png}
%     \caption{Biểu đồ so sánh độ trễ giữa HNSW và Brute-force}
%     \label{fig:system_architecture}
% \end{figure}


% Kết quả cho thấy các thuật toán dựa trên HNSW cải thiện đáng kể tốc độ truy vấn so với phương pháp vét cạn (Brute-force). Cụ thể, thuật toán HNSW-PQ cho hiệu năng tốt nhất với độ trễ trung bình khoảng 98 µs, nhanh gấp 6 lần so với Brute-force (~590 µs). Các biến thể khác như HNSWLib, HNSW-Flat và HNSW-SQ cũng duy trì độ trễ ổn định trong khoảng 200-260 µs, nhanh hơn Brute-force từ 2.2 đến 2.8 lần.

% \subsection{Độ chính xác (Accuracy)}
% Độ chính xác được đo bằng Recall@K, được định nghĩa là tỷ lệ các kết quả từ HNSW xuất hiện trong top-K kết quả chính xác từ brute-force. Kết quả được trình bày trong \Cref{tab:recall_comparison}.

% \begin{table}[htbp]
%     \centering
%     \caption{Độ chính xác Recall@K của HNSW}
%     \label{tab:recall_comparison}
%     \begin{tabular}{@{}ccccc@{}}
%         \toprule
%         \textbf{K} & & \textbf{Recall@K} & & \\
%          & \textbf{HNSWLib} & \textbf{HNSW-Flat} & \textbf{HNSW-SQ} & \textbf{HNSW-PQ} \\
%         \midrule
%         1   & 0.9800  & 0.9800  & 0.9500  & 0.1700  \\
%         5   & 0.9680  & 0.9800  & 0.9640  & 0.2560  \\
%         10  & 0.9670  & 0.9740  & 0.9620  & 0.2900  \\
%         20  & 0.9625  & 0.9695  & 0.9600  & 0.3110  \\
%         50  & 0.9412  & 0.9508  & 0.9454  & 0.3318  \\
%         100 & 0.9196  & 0.9315  & 0.9277  & 0.3565  \\
%         \bottomrule
%     \end{tabular}
% \end{table}

% \begin{figure}[htbp]
%     \centering
%     \includegraphics[width=1\textwidth]{images/recallK.png}
%     \caption{Biểu đồ so sánh độ chính xác Recall@K của HNSW}
%     \label{fig:system_architecture}
% \end{figure}

% \begin{itemize}
%     \item \textbf{Độ chính xác cao nhất (HNSWLib \& HNSW-Flat):} Hai đường biểu diễn nằm sát đường cơ sở (\textit{Brute Force}) với Recall@1 đạt $0.98$. Đây là lựa chọn tối ưu khi ưu tiên độ chính xác tuyệt đối và không bị giới hạn về bộ nhớ RAM.

%     \item \textbf{Cân bằng tối ưu (HNSW-SQ):} Mặc dù sử dụng lượng tử hóa vô hướng (\textit{Scalar Quantization}) để nén dữ liệu, độ chính xác vẫn rất cao (Recall@1 đạt $0.95$), chỉ thấp hơn bản Flat không đáng kể. Đây là phương pháp hiệu quả nhất xét trên tỷ lệ \textit{hiệu năng/tài nguyên}.

%     \item \textbf{Hiệu năng thấp (HNSW-PQ):} Kỹ thuật \textit{Product Quantization} trong trường hợp này làm mất mát quá nhiều thông tin, dẫn đến độ chính xác rất thấp (Recall@1 chỉ đạt $0.17$). Không khuyến nghị sử dụng cấu hình này cho tập dữ liệu hiện tại.
% \end{itemize}

% \noindent \textbf{Kết luận:} Dựa trên biểu đồ, \textbf{HNSW-SQ} là giải pháp tốt nhất để triển khai thực tế nhờ khả năng tiết kiệm bộ nhớ trong khi vẫn duy trì độ chính xác cao (~$96\%$).

% \subsection{Khả năng mở rộng (Scalability)}
% Khả năng mở rộng được đánh giá bằng cách đo thời gian xây dựng chỉ mục và thời gian truy vấn khi tăng số lượng vector. Kết quả được trình bày trong \Cref{fig:scalability}.

% \subsection{Ảnh hưởng của tham số M}
% \Cref{fig:m_parameter} minh họa mối quan hệ giữa tham số M và hiệu suất (độ chính xác và thời gian truy vấn). Biểu đồ cho thấy có điểm tối ưu ở M=200, nơi đạt được sự cân bằng tốt giữa độ chính xác và tốc độ.

% \begin{figure}[htbp]
%     \centering
%     \includegraphics[width=0.8\textwidth]{images/m_parameter.png}
%     \caption{Ảnh hưởng của tham số M đến hiệu suất}
%     \label{fig:m_parameter}
% \end{figure}

% \subsection{Heatmap điều chỉnh tham số}
% \Cref{fig:parameter_heatmap} trình bày heatmap so sánh hiệu suất với các cấu hình tham số khác nhau (M vs efConstruction). Màu sắc đại diện cho giá trị Recall@1, với màu đỏ đậm hơn biểu thị độ chính xác cao hơn.

% \begin{figure}[htbp]
%     \centering
%     \includegraphics[width=0.8\textwidth]{images/parameter_heatmap.png}
%     \caption{Heatmap điều chỉnh tham số (M vs efConstruction)}
%     \label{fig:parameter_heatmap}
% \end{figure}

% \section{Cấu trúc đồ thị}

% \subsection{Visualization phân tầng}
% \Cref{fig:hnsw_layers} minh họa cấu trúc phân tầng của đồ thị HNSW. Hình vẽ cho thấy:
% \begin{itemize}
%     \item Tầng 0 (dưới cùng): Mật độ nút cao nhất, tất cả các vector đều có mặt
%     \item Các tầng cao hơn: Mật độ giảm dần, chỉ một phần nhỏ vector xuất hiện
%     \item Tầng cao nhất: Rất thưa, chỉ có một vài nút làm điểm vào
% \end{itemize}

% \begin{figure}[htbp]
%     \centering
%     \includegraphics[width=0.8\textwidth]{images/hnsw_layers.png}
%     \caption{Cấu trúc phân tầng của đồ thị HNSW}
%     \label{fig:hnsw_layers}
% \end{figure}

% \subsection{Topology đồ thị}
% \Cref{fig:graph_topology} trình bày topology của đồ thị HNSW ở một tầng cụ thể, cho thấy các kết nối giữa các nút. Mỗi nút được kết nối với M nút gần nhất, tạo thành một mạng lưới nhỏ thế giới (small world network).

% \begin{figure}[htbp]
%     \centering
%     \includegraphics[width=0.8\textwidth]{images/graph_topology.png}
%     \caption{Topology đồ thị HNSW ở một tầng}
%     \label{fig:graph_topology}
% \end{figure}

% \section{Quá trình tìm kiếm}

% \subsection{Đường đi tìm kiếm}
% \Cref{fig:search_path} minh họa đường đi của thuật toán greedy search từ điểm vào (màu đỏ) đến vector mục tiêu (màu xanh lá). Đường đi cho thấy:
% \begin{itemize}
%     \item Bắt đầu từ tầng cao nhất
%     \item Di chuyển xuống các tầng thấp hơn
%     \item Tại mỗi tầng, chọn nút gần nhất với truy vấn
%     \item Dừng lại khi không tìm thấy nút gần hơn
% \end{itemize}

% \begin{figure}[htbp]
%     \centering
%     \includegraphics[width=0.8\textwidth]{images/search_path.png}
%     \caption{Đường đi tìm kiếm trong đồ thị HNSW}
%     \label{fig:search_path}
% \end{figure}

% \subsection{Visualization 3D}
% \Cref{fig:hnsw_3d} trình bày visualization 3D của đồ thị HNSW, cho phép quan sát cấu trúc phân tầng và các kết nối từ nhiều góc độ khác nhau. Visualization này giúp hiểu rõ hơn về cách đồ thị được tổ chức trong không gian nhiều chiều.

% \begin{figure}[htbp]
%     \centering
%     \includegraphics[width=0.8\textwidth]{images/hnsw_3d.png}
%     \caption{Visualization 3D của đồ thị HNSW}
%     \label{fig:hnsw_3d}
% \end{figure}

% \section{So sánh đa phương thức}

% \subsection{Hiệu suất theo loại dữ liệu}
% \Cref{fig:multimodal_comparison} so sánh hiệu suất của HNSW trên ba loại dữ liệu khác nhau:
% \begin{itemize}
%     \item \textbf{Hình ảnh} (512 chiều): Độ trễ ~45µs, Recall@10 ≈ 0.85
%     \item \textbf{Tài liệu} (1024 chiều): Độ trễ ~60µs, Recall@10 ≈ 0.82
%     \item \textbf{Y tế} (512 chiều): Độ trễ ~40µs, Recall@10 ≈ 0.88
% \end{itemize}

% \begin{figure}[htbp]
%     \centering
%     \includegraphics[width=0.8\textwidth]{images/multimodal_comparison.png}
%     \caption{So sánh hiệu suất trên các loại dữ liệu khác nhau}
%     \label{fig:multimodal_comparison}
% \end{figure}

% \subsection{Ảnh hưởng của số chiều}
% \Cref{fig:dimension_impact} minh họa ảnh hưởng của số chiều embedding đến hiệu suất. Kết quả cho thấy:
% \begin{itemize}
%     \item Số chiều cao hơn → thời gian truy vấn tăng nhẹ
%     \item Số chiều không ảnh hưởng đáng kể đến độ chính xác
%     \item HNSW hoạt động hiệu quả với cả embedding 512 và 1024 chiều
% \end{itemize}

% \begin{figure}[htbp]
%     \centering
%     \includegraphics[width=0.8\textwidth]{images/dimension_impact.png}
%     \caption{Ảnh hưởng của số chiều embedding đến hiệu suất}
%     \label{fig:dimension_impact}
% \end{figure}


