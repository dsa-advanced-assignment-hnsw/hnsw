\chapter{Trực quan hóa thuật toán}

\section{Mẫu trực quan hóa}

Các thí nghiệm được thực hiện trên nền tảng đám mây Google Colab (phiên bản miễn phí) với cấu hình chi tiết như sau:
\begin{itemize}
    \item \textbf{Số lượng phần tử}: 20
    \item \textbf{M (Số liên kết tối đa)}: 3
    \item \textbf{efConstruction}: 3
    \item \textbf{efSearch}: 5
    \item \textbf{k (số láng giềng gần nhất)}: 10
\end{itemize}

\section{Cấu trúc đồ thị 3D}
\Cref{fig:hnsw_3d} minh họa cấu trúc không gian 3 chiều của đồ thị HNSW, cung cấp cái nhìn trực quan về sự phân tầng và mạng lưới liên kết giữa các vector. Việc trực quan hóa này giúp làm rõ cách thức tổ chức dữ liệu trong không gian nhiều chiều:
\begin{itemize}
    \item \textbf{Tầng cơ sở (Layer 0):} Có mật độ nút cao nhất, chứa toàn bộ các vector dữ liệu.
    \item \textbf{Các tầng trung gian:} Mật độ giảm dần, đóng vai trò cầu nối giúp thuật toán định hướng nhanh.
    \item \textbf{Tầng cao nhất:} Cấu trúc thưa thớt, chứa duy nhất nút điểm vào (entry point - nút màu đỏ).
\end{itemize}

\begin{figure}[H]
    \centering
    \includegraphics[width=0.8\textwidth]{images/visualize_hnsw_light.png}
    \caption{Mô hình 3D thể hiện sự phân tầng của HNSW}
    \label{fig:hnsw_3d}
\end{figure}

\section{Mô phỏng quá trình tìm kiếm}
\Cref{fig:search_path} minh họa trực quan lộ trình di chuyển của giải thuật Greedy Search trong không gian đồ thị phân tầng. Quá trình này mô phỏng cơ chế "zoom-in", cho phép hệ thống thu hẹp nhanh chóng vùng tìm kiếm từ quy mô toàn cục xuống cục bộ. Các giai đoạn thực thi cụ thể bao gồm:

\begin{itemize}
    \item \textbf{Khởi tạo (Initialization):} Thuật toán bắt đầu tại điểm vào (entry point - nút màu đỏ) nằm ở tầng cao nhất, nơi đồ thị thưa thớt nhất để thực hiện các bước nhảy lớn.
    \item \textbf{Duyệt tầng (Layer Navigation):} Tại mỗi tầng, thuật toán thực hiện tìm kiếm tham lam bằng cách so sánh khoảng cách và chọn nút láng giềng có độ tương đồng cao nhất với vector truy vấn (query vector).
    \item \textbf{Chuyển tầng (Hierarchical Descent):} Khi không còn láng giềng nào gần truy vấn hơn nút hiện tại, thuật toán hạ xuống tầng tiếp theo ngay tại vị trí đó để tinh chỉnh kết quả.
    \item \textbf{Kết thúc (Convergence):} Quy trình lặp lại cho đến khi đạt trạng thái tối ưu cục bộ tại tầng cơ sở (Layer 0), trả về vector mục tiêu (màu xanh lá) là kết quả tìm kiếm gần đúng nhất.
\end{itemize}

\begin{figure}[H]
    \centering
    \includegraphics[width=0.8\textwidth]{images/visualize_search_light.png}
    \caption{Đường dẫn tìm kiếm vector mục tiêu trong không gian HNSW}
    \label{fig:search_path}
\end{figure}

\section{Cơ chế thêm phần tử mới}
\Cref{fig:insert_hnsw} minh họa chi tiết quy trình cập nhật cấu trúc đồ thị khi chèn một phần tử mới (nút màu vàng)[cite: 1487]. Tầng cao nhất ($l$) của phần tử này được xác định dựa trên hàm phân phối xác suất cấp độ, giúp duy trì cấu trúc phân tầng tối ưu cho đồ thị. Quy trình thiết lập liên kết được thực hiện qua các giai đoạn sau:

\begin{itemize}
    \item \textbf{Tìm kiếm láng giềng (Neighbor Search):} Tại mỗi tầng từ $l$ xuống tầng 0, thuật toán thực hiện tìm kiếm tập hợp $M$ láng giềng gần nhất với nút mới thông qua giải thuật tham lam.
    \item \textbf{Thiết lập liên kết (Connection Establishment):} Các cạnh hai chiều mới (màu xanh) được khởi tạo để tích hợp phần tử mới vào mạng lưới hiện tại, đảm bảo tính kết nối và khả năng điều hướng của đồ thị.
    \item \textbf{Tối ưu hóa cấu trúc (Heuristic Pruning):} Trong trường hợp số lượng kết nối của một nút vượt quá ngưỡng $M_{max}$, thuật toán sẽ kích hoạt cơ chế tái cấu trúc (SELECT-NEIGHBORS-HEURISTIC). 
    \item \textbf{Cắt tỉa cạnh (Edge Pruning):} Thuật toán thực hiện đánh giá lại các liên kết dựa trên cả khoảng cách và sự đa dạng về hướng, sau đó loại bỏ các cạnh thừa (màu đỏ) nhằm kiểm soát độ phức tạp bộ nhớ và duy trì hiệu suất truy vấn logarit.
\end{itemize}

\begin{figure}[H]
    \centering
    \includegraphics[width=0.8\textwidth]{images/visualize_insert_light.png}
    \caption{Quá trình thêm nút và cập nhật cạnh trong HNSW}
    \label{fig:insert_hnsw}
\end{figure}