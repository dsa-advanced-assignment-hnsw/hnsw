\chapter*{Lời nói đầu}
Từ thuở sơ khai của máy tính cá nhân và Internet, các hệ thống tìm kiếm (search engine) luôn là một ứng dụng quan trọng khi người dùng có nhu cầu tìm kiếm thứ gì đó trên Internet, với đại diện tiêu biểu mà phần lớn người dùng trên thế giới đều trải nghiệm qua, đó là Google. Hay Youtube, không phải một cách ngẫu nhiên mà Youtube luôn cho ra những video đề xuất đúng với thứ mà người dùng cần tìm khi nhập vào thanh tìm kiếm. Từ những phép so sánh chuỗi giống nhau để xuất ra kết quả tìm kiếm, người ta bắt đầu sử dụng các kĩ thuật hiện đại hơn để dễ dàng "hiểu ý" người dùng. Từ đó mà cơ sở dữ liệu vector (vector database) ra đời.

Ngày nay, search engines đóng vai trò không nhỏ trong đời sống của mỗi cá nhân. Các hệ thống này càng ngày càng hiểu ý người dùng và đưa cho ta những câu trả lời cho những gì mà ta nhập vào. Các thuật toán tìm kiếm tương tự mới (similarity search algorithms) tương tác với vector database cũng ra đời để làm cho truy vấn của người dùng, trong thời gian ngắn nhất, đến được với những dữ liệu gần với truy vấn nhất. Cùng với sự phát triển của trí tuệ nhân tạo (Artificial Intelligence - AI) mà trên thị trường, doanh nghiệp nào có các thuật toán càng hiện đại, càng nhanh, càng chính xác, thì sẽ càng thu hút người dùng và càng có được nguồn doanh thu càng lớn.

Thấy được tầm quan trọng và ứng dụng vào doanh nghiệp của các thuật toán tìm kiếm tương tự, cụ thể ở đây là thuật toán dựa trên cấu trúc đồ thị HNSW, nhóm sinh viên chọn đề tài này để nghiên cứu và phát triển.

\textit{Nhóm sinh viên.}
\chapter*{Lời cảm ơn}
Nhóm sinh viên gửi lời cảm ơn sâu sắc đến giảng viên hướng dẫn - TS.Lê Thành Sách, vì đã truyền tải những kiến thức quý báu về các vấn đề liên quan và hỗ trợ nhóm thực hiện bài báo cáo cho học phần mở rộng này.

Cảm ơn ban lãnh đạo Trường Đại học Bách khoa - ĐHQG TP.HCM vì đã đưa học phần mở rộng vào chương trình tài năng của các sinh viên, góp phần rất lớn giúp nhóm sinh viên nâng cao các kĩ năng chuyên môn để áp dụng cho nghề nghiệp sau này.

Cảm ơn tập thể các sinh viên chương trình tài năng cùng lớp trong học phần mở rộng vì đã đưa ra các phản biện quý báu, qua đó giúp cho bài báo cáo của nhóm sinh viên hoàn thiện hơn.

Nhóm sinh viên ý thức rằng, với kinh nghiệm còn hạn chế và kiến thức chưa sâu rộng, bài làm của nhóm chắc chắn không tránh khỏi những thiếu sót. Rất mong nhận được những ý kiến đóng góp, nhận xét quý báu từ giảng viên hướng dẫn và bạn đọc để báo cáo của nhóm được hoàn thiện hơn.

\textit{Nhóm sinh viên.}
